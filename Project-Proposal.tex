% Options for packages loaded elsewhere
\PassOptionsToPackage{unicode}{hyperref}
\PassOptionsToPackage{hyphens}{url}
%
\documentclass[
]{article}
\title{Project Proposal}
\usepackage{etoolbox}
\makeatletter
\providecommand{\subtitle}[1]{% add subtitle to \maketitle
  \apptocmd{\@title}{\par {\large #1 \par}}{}{}
}
\makeatother
\subtitle{Interactive Shiny App: Carbon Emissions by Coutry and by Year}
\author{from Daniel Welz, Guillaume Bilocq, Jasmine Mawjee, Virany Kho,
Colin Steffe}
\date{2021-11-07}

\usepackage{amsmath,amssymb}
\usepackage{lmodern}
\usepackage{iftex}
\ifPDFTeX
  \usepackage[T1]{fontenc}
  \usepackage[utf8]{inputenc}
  \usepackage{textcomp} % provide euro and other symbols
\else % if luatex or xetex
  \usepackage{unicode-math}
  \defaultfontfeatures{Scale=MatchLowercase}
  \defaultfontfeatures[\rmfamily]{Ligatures=TeX,Scale=1}
\fi
% Use upquote if available, for straight quotes in verbatim environments
\IfFileExists{upquote.sty}{\usepackage{upquote}}{}
\IfFileExists{microtype.sty}{% use microtype if available
  \usepackage[]{microtype}
  \UseMicrotypeSet[protrusion]{basicmath} % disable protrusion for tt fonts
}{}
\makeatletter
\@ifundefined{KOMAClassName}{% if non-KOMA class
  \IfFileExists{parskip.sty}{%
    \usepackage{parskip}
  }{% else
    \setlength{\parindent}{0pt}
    \setlength{\parskip}{6pt plus 2pt minus 1pt}}
}{% if KOMA class
  \KOMAoptions{parskip=half}}
\makeatother
\usepackage{xcolor}
\IfFileExists{xurl.sty}{\usepackage{xurl}}{} % add URL line breaks if available
\IfFileExists{bookmark.sty}{\usepackage{bookmark}}{\usepackage{hyperref}}
\hypersetup{
  pdftitle={Project Proposal},
  pdfauthor={from Daniel Welz, Guillaume Bilocq, Jasmine Mawjee, Virany Kho, Colin Steffe},
  hidelinks,
  pdfcreator={LaTeX via pandoc}}
\urlstyle{same} % disable monospaced font for URLs
\usepackage[margin=1in]{geometry}
\usepackage{graphicx}
\makeatletter
\def\maxwidth{\ifdim\Gin@nat@width>\linewidth\linewidth\else\Gin@nat@width\fi}
\def\maxheight{\ifdim\Gin@nat@height>\textheight\textheight\else\Gin@nat@height\fi}
\makeatother
% Scale images if necessary, so that they will not overflow the page
% margins by default, and it is still possible to overwrite the defaults
% using explicit options in \includegraphics[width, height, ...]{}
\setkeys{Gin}{width=\maxwidth,height=\maxheight,keepaspectratio}
% Set default figure placement to htbp
\makeatletter
\def\fps@figure{htbp}
\makeatother
\setlength{\emergencystretch}{3em} % prevent overfull lines
\providecommand{\tightlist}{%
  \setlength{\itemsep}{0pt}\setlength{\parskip}{0pt}}
\setcounter{secnumdepth}{-\maxdimen} % remove section numbering
\usepackage{booktabs}
\usepackage{longtable}
\usepackage{array}
\usepackage{multirow}
\usepackage{wrapfig}
\usepackage{float}
\usepackage{colortbl}
\usepackage{pdflscape}
\usepackage{tabu}
\usepackage{threeparttable}
\usepackage{threeparttablex}
\usepackage[normalem]{ulem}
\usepackage{makecell}
\usepackage{xcolor}
\ifLuaTeX
  \usepackage{selnolig}  % disable illegal ligatures
\fi

\begin{document}
\maketitle

\begin{center}\rule{0.5\linewidth}{0.5pt}\end{center}

\hypertarget{goals-and-impact}{%
\section{Goals and Impact}\label{goals-and-impact}}

All of us are aware of the urgent need to reduce our individual
ecological footprint by reducing our overall consumption, shift our
purchasing habits, recycle more, invest in renewable energies and
globally contribute to a common goal: a cleaner planet. Even though
world leaders gather to unify and work as a united front towards
sustainable changes with the ultimate goal to slow climate change, we
still currently detect many problems surrounding the topic of
sustainability.

First, we understand that many aspects of sustainability are not well
understood. Indeed, it is associated with many difficult terms (e.g.~bio
capacity, life-cycle analysis, embodied energy, footprint intensity,
etc.) that non-specialists do not necessarily understand. Because, the
average layman does not necessarily understand these terms and is
sometimes overwhelmed by the overload of information in the News
regarding achieving sustainability, individuals tend to completely turn
away from thinking about it.

Second, it is not immediately obvious how our personal immediate actions
(e.g.~limit water use waste, riding a bike over driving a car) fit the
overall picture of the community, country and the world.

Third, sustainability data is scattered among all kinds of media in a
``fuzzy'' way. Oftentimes, when doing research about the topic, one ends
up being constantly redirected to more and more sources that are not
necessarily mutually exclusive and collectively exhaustive.

To address the above points and challenges, we want to create the
S(ust)hinybility app based on a reliable database. This app aims at
gathering worldwide data on carbon emissions caused by different types
of production lands and exploit the information to produce visual
outcomes enabling a user to look for specific information by country and
by year. Therefore, the shiny app will provide visualization to compare
carbon emissions on a country (or region) level and understand the
gravity and urgency of undertaking immediate long-lasting actions
against climate change.

We try to have the impact on three different levels. First, the
individuals are nudged to learn about ecological footprints. This,
increases knowledge and awareness and helps to make more informed
decisions later on. Second, making ecological footprints comparable
between countries helps to filter certain criteria and makes research
more convenient. Here, the impact is a higher satisfaction of
researchers looking for countries' involvement and responsibility
towards sustainable changes. Third, the interactive app helps to track
and trace developments. This tool could be a fact checker and used
against wrongly spread information over the Internet in either way.

\hypertarget{plan}{%
\section{Plan}\label{plan}}

Our approach is to explore a data set of more than 70,000 entries to
investigate carbon emissions by country (and region) over time and,
provide some visualization (\emph{map}, \emph{ggplot2} packages). To
investigate the evolution over time of CO2 emissions, we will use time
series (\emph{tsibble} package) and ultimately, we will forecast
(\emph{package FPP3}) emissions for an horizon of time t, to be
determined. To explore the data set and provide several features to our
shiny app, we will use packages enabling us to manipulate the data
(\emph{dplyr}, \emph{tidyverse}) to add features using built-in and our
own functions before visually present (\emph{map}, \emph{ggplot2}) our
findings.

In order to insure that the app is interactive, we will use features
such as checkboxes, filter buttons, flags to let the user be as active
as possible.

Our source of inspiration came from one of our brainstorming sessions
about the semester project. Indeed, we wanted to create a shiny app
providing information on our carbon footprint. Unfortunately, we
couldn't find nor create (lack of time) a data set providing us with the
exact information needed. Therefore, we decided to focus on countries'
carbon emissions since 1961 and study the magnitude differences between
them regarding certain criteria.

It is very likely that such application is already existing but our
focus is to apply notions learnt in class to this specific data set
through an interactive shiny app enabling the users to select different
features (e.g.~checkboxes, filter button\ldots) and have a customized
output.

Our data collection strategy was to look for official websites providing
reliable data on carbon emissions by country. We chose this approach to
rely on well collected and organized data.

We found and downloaded a data set containing more than 70,000 entries
ordered by country and by year on the following website:
\url{https://data.world/footprint}. We uploaded the data set in r and
saved in the \textbf{read\_data.R} file. Moreover we might enrich this
data with other variables such as population, GDP or others.

\hypertarget{timeline-and-milestones}{%
\section{Timeline and Milestones}\label{timeline-and-milestones}}

\includegraphics{Project-Proposal_files/figure-latex/Gant chart-1.pdf}

\hypertarget{management-plan}{%
\section{Management Plan}\label{management-plan}}

For the purpose of the progress of our project, we are planning to have
at least one Zoom meeting per week, usually on Fridays, to 1) track our
advancements 2) discuss and solve potential problems 3) decide on our
next steps and 4) split tasks 5) set weekly goals.

At each meeting, each team member will be able to share personal
advancements and achievements on their respective tasks, ask for help if
needed and share potential improvement ideas for other tasks. Each team
member will be assigned to one or more tasks depending on tasks'
difficulty level. Indeed, a task can have few people working on it if it
is time-consuming.

\emph{Table 1} displays the mandatory administrative tasks that we
already assigned to team members.

\begin{table}

\caption{\label{tab:administrative tasks}Group members administrative tasks}
\centering
\resizebox{\linewidth}{!}{
\begin{tabular}[t]{cc}
\toprule
Gp\_members & Admin\_tasks\\
\midrule
Colin & will be reminding everyone about the Zoom meeting every week, create it and send it on the group chat\\
Jasmine & will track the advancement of the project on Github, manage the organization of the Project folder so that it does not become messy as the project progress\\
Guillaume & will manage the project advancement in order to not be late compared to the timeline we decided to set for each task\\
Daniel & will make sure the Github repository is well-documented with nice commit messages, with clear issues and that the To Do list is being updated along the progression of the project\\
Virany & will verify that the overall code is well-written, with a good coding style and good commentary to explain it\\
\bottomrule
\end{tabular}}
\end{table}

Concerning the tasks related to the actual content of the project, we
will assign them after deepening our knowledge and familiarizing with
the data we found.

\end{document}
